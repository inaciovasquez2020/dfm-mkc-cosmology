% main.tex (compile-ready)
% DFM–MKC Cosmology: minimal kinetic-coupling dark-fluid mechanism for DESI-era dynamical DE preference
% Target venues: PRD (REVTeX 4-2) or (with minor edits) JCAP.

\documentclass[reprint,aps,prd,nofootinbib,superscriptaddress]{revtex4-2}

% ---------- Packages ----------
\usepackage{amsmath,amssymb,mathtools}
\usepackage{bm}
\usepackage{graphicx}
\usepackage{hyperref}
\usepackage{booktabs}
\usepackage{siunitx}
\usepackage{xspace}

% ---------- Macros ----------
\newcommand{\LCDM}{\ensuremath{\Lambda\mathrm{CDM}}\xspace}
\newcommand{\wde}{\ensuremath{w_{\rm DE}}\xspace}
\newcommand{\wz}{\ensuremath{w(z)}\xspace}
\newcommand{\weff}{\ensuremath{w_{\rm eff}}\xspace}
\newcommand{\H0}{\ensuremath{H_0}\xspace}
\newcommand{\S8}{\ensuremath{S_8}\xspace}
\newcommand{\Om}{\ensuremath{\Omega_{\rm m}}\xspace}
\newcommand{\Ode}{\ensuremath{\Omega_{\rm DE}}\xspace}
\newcommand{\Oc}{\ensuremath{\Omega_{\rm c}}\xspace}
\newcommand{\Ob}{\ensuremath{\Omega_{\rm b}}\xspace}

\begin{document}

\title{Minimal Kinetic-Coupling Dark Fluid Cosmology:\\
A Mechanism-Level Interpretation of DESI-Era Preference for Dynamical Dark Energy}

\author{Inacio F. Vasquez}
\affiliation{Independent Researcher}

\date{December 22, 2025}

\begin{abstract}
Recent joint analyses combining DESI-era baryon acoustic oscillation measurements with external datasets
(e.g., CMB and Type Ia supernova compilations) report a dataset- and prior-dependent preference for
dynamical dark energy relative to a constant cosmological constant, typically expressed through the reconstructed evolution of \wz.
Motivated by these developments, we present a minimal kinetic-coupling dark-fluid (DFM--MKC) framework in which
energy--momentum exchange within the dark sector reproduces effective quintom-like behavior (including permitted crossings of $\wde=-1$ in the effective description)
without introducing a universal fifth force.
We derive the background and linear-perturbation equations, define compact coupling parameterizations that isolate the phenomenology relevant to DESI-era constraints,
and formulate mechanism-level consistency relations intended to distinguish dark-sector coupling from purely phenomenological \wz reconstructions.
We outline a falsification program spanning expansion history, growth of structure, and CMB lensing, and we provide an analysis roadmap for joint likelihood evaluation using DESI BAO + CMB + SN(+WL),
with explicit robustness checks under alternative priors, dataset subsets, and coupling parameterizations.
\end{abstract}

\maketitle

% =========================================================
\section{Introduction}
\label{sec:intro}

The \LCDM model remains an extraordinarily successful baseline description of cosmological observations.
Nevertheless, late-time probes continue to motivate extensions that relax the assumption of a strictly constant dark-energy equation of state.
DESI-era BAO measurements sharpen constraints on $D_M(z)$ and $H(z)$, and multiple joint analyses that combine BAO with CMB and supernova data report a preference for dynamical dark energy relative to \LCDM,
with the reported strength depending on dataset choices, prior ranges, and the parameterization used for \wz.

\paragraph{What is (and is not) claimed.}
We distinguish: (i) \emph{phenomenological preference} for non-constant \wz in certain joint analyses, from (ii) \emph{mechanism identification} via a physical model that explains the preference and survives cross-probe consistency tests.
This work supplies a minimal mechanism-level candidate---DFM--MKC---together with falsifiable signatures,
without asserting that current data uniquely confirms the mechanism.

\paragraph{Contributions.}
We provide:
\begin{enumerate}
  \item A minimal interacting dark-sector kinetic-coupling model with a controlled separation between background expansion and growth-of-structure effects.
  \item An explicit mapping between coupling parameters and the effective equation-of-state history \weff$(z)$, including conditions for effective phantom crossing.
  \item Mechanism-level consistency relations and a falsification checklist designed for DESI-era joint analyses.
\end{enumerate}

% =========================================================
\section{Model: Minimal Kinetic-Coupling Dark Fluid (DFM--MKC)}
\label{sec:model}

\subsection{Fluid-level interacting dark sector}

We assume a spatially flat FLRW background with scale factor $a(t)$ and Hubble rate $H\equiv \dot a/a$.
The dark sector is composed of pressureless cold dark matter (DM) with density $\rho_{\rm c}$ and dark energy (DE) with density $\rho_{\rm de}$ and pressure $p_{\rm de}$.
Energy exchange within the dark sector is modeled as
\begin{align}
\dot\rho_{\rm c} + 3H\rho_{\rm c} &= +Q, \label{eq:dm_cont}\\
\dot\rho_{\rm de} + 3H(\rho_{\rm de}+p_{\rm de}) &= -Q, \label{eq:de_cont}
\end{align}
with $Q$ defining the direction of energy flow:
$Q>0$ corresponds to DE$\to$DM and $Q<0$ to DM$\to$DE.

\subsection{Minimal kinetic-coupling ansatz}

A minimal DESI-era--testable choice is
\begin{equation}
Q = 3H\,\xi(a)\,\rho_{\rm c},
\label{eq:Q_def}
\end{equation}
where $\xi(a)$ is a (small) dimensionless coupling function.

Two practical parameterizations are:
\begin{align}
\xi(a) &= \xi_0 + \xi_a (1-a), \label{eq:xi_cpl}\\
\xi(a) &= \xi_0\,\frac{1+\tanh\!\left(\frac{a-a_t}{\Delta}\right)}{2}
        - \xi_1\,\frac{1-\tanh\!\left(\frac{a-a_t}{\Delta}\right)}{2},
\label{eq:xi_switch}
\end{align}
where Eq.~\eqref{eq:xi_switch} permits a sign reversal around $a_t$ if preferred by joint fits.

\paragraph{Intrinsic DE equation of state.}
For minimality, we take a constant intrinsic equation of state,
\begin{equation}
p_{\rm de} = w_0\,\rho_{\rm de},
\label{eq:w0}
\end{equation}
and allow effective evolution to arise from coupling via $Q$.
(Extensions with a CPL intrinsic form can be treated as a controlled generalization.)

\subsection{Effective equation of state and effective phantom crossing}

Define the effective DE equation of state inferred from background evolution as
\begin{equation}
\weff(a) \equiv -1 - \frac{1}{3}\frac{d\ln\rho_{\rm de}}{d\ln a}.
\label{eq:weff_def}
\end{equation}
Using Eq.~\eqref{eq:de_cont}, one obtains
\begin{equation}
\weff(a) = w_0 + \frac{Q}{3H\rho_{\rm de}}
         = w_0 + \xi(a)\,\frac{\rho_{\rm c}}{\rho_{\rm de}}.
\label{eq:weff_Q}
\end{equation}
Hence $\weff$ can cross $-1$ depending on the sign and magnitude of $\xi(a)$ even if $w_0>-1$,
enabling effective ``phantom crossing'' without introducing a fundamental phantom field.
Stability conditions depend on the microphysical completion (fluid closure / EFT / action-level model); in this paper we adopt the conservative requirement that the chosen perturbation prescription is ghost- and gradient-stable in the parameter region explored (Appendix~\ref{app:stability}).

% =========================================================
\section{Background Dynamics and Parameter Degeneracies}
\label{sec:background}

\subsection{Friedmann equations}

In a flat universe,
\begin{equation}
H^2(a) = H_0^2\Big[\Omega_{\rm r}a^{-4} + \Omega_{\rm m}a^{-3} + \Omega_{\rm de}(a)\Big],
\end{equation}
where $\Omega_{\rm m}=\Omega_{\rm b}+\Omega_{\rm c}$ and radiation/neutrinos may be included as needed.

Given \weff$(a)$, the DE density evolves as
\begin{equation}
\rho_{\rm de}(a) = \rho_{\rm de,0}\,
\exp\!\left(3\int_a^1 \frac{1+\weff(a')}{a'}\,da'\right).
\label{eq:rde}
\end{equation}
Equations~\eqref{eq:dm_cont}--\eqref{eq:de_cont} with $Q$ from Eq.~\eqref{eq:Q_def} provide a closed background system for $(\rho_{\rm c},\rho_{\rm de})$ given $(w_0,\xi)$.

\subsection{Mapping to DESI-era constraints}

BAO primarily constrains the comoving angular diameter distance $D_M(z)$ and the Hubble rate $H(z)$ via the BAO scale.
In interacting models, $(w_0,\xi)$ can be partially degenerate in background-only fits, since both modify the late-time expansion history through $\weff(a)$.
This motivates: (i) explicit robustness checks across dataset subsets; and (ii) incorporating growth and lensing observables that respond differently to coupling than to purely phenomenological \wz reconstructions.

% =========================================================
\section{Linear Perturbations and Growth of Structure}
\label{sec:perturbations}

\subsection{Perturbation prescription and schematic growth equation}

Interacting dark-sector models require a specified covariant prescription for momentum transfer and gauge conventions.
We adopt a standard choice in which the energy--momentum transfer four-vector is aligned with the DM four-velocity at the background level; details and conventions are given in Appendix~\ref{app:stability}.
On sub-horizon scales, the growth of DM density perturbations can be expressed schematically as
\begin{equation}
\delta_{\rm c}'' + \left(2 + \frac{H'}{H} + \Gamma(a)\right)\delta_{\rm c}'
- \frac{3}{2}\Omega_{\rm m}(a)\,\mu(a)\,\delta_{\rm c} = 0,
\label{eq:growth_schematic}
\end{equation}
where primes denote derivatives with respect to $\ln a$ and $\Gamma(a),\mu(a)$ encode effective drag and clustering modifications induced by coupling.
In DFM--MKC these functions are determined by $\xi(a)$ and the chosen interaction prescription and can be extracted numerically.

\subsection{Connection to \S8 and lensing}

The coupling modifies growth observables such as $f\sigma_8(z)$, weak-lensing summaries (including \S8), and the CMB lensing potential power spectrum.
A central goal is to identify signatures that are diagnostic of coupling \emph{as a mechanism} rather than generic \wz phenomenology, e.g. correlated residual patterns between growth-rate observables and lensing that track the inferred coupling transition scale $a_t$ in Eq.~\eqref{eq:xi_switch}.

% =========================================================
\section{Data, Likelihood Strategy, and Robustness Checks}
\label{sec:data}

\subsection{Datasets}

The intended joint analyses include:
\begin{itemize}
\item DESI-era BAO measurements (specify sample selection and redshift bins as used in the likelihood).
\item CMB constraints (e.g., Planck 2018 baseline; optional ACT/SPT combinations as robustness).
\item Type Ia supernovae (e.g., Pantheon+ or an explicitly stated alternative compilation).
\item Optional: weak lensing (DES/KiDS/HSC; and later Euclid/Roman updates when available).
\end{itemize}

\subsection{Inference pipeline}

We consider parameter vectors of the form
\begin{equation}
\theta = (\Omega_{\rm b}h^2,\Omega_{\rm c}h^2, H_0, n_s, A_s, \tau, \ldots, w_0, \text{coupling parameters}),
\end{equation}
with coupling parameters chosen as $(\xi_0,\xi_a)$ or $(\xi_0,\xi_1,a_t,\Delta)$.
We report posterior stability under alternative priors and evaluate model comparison (e.g., Bayes factors $\ln B$) with explicit prior-dependence accounting.

\subsection{Robustness and falsification checklist}

We adopt the following non-negotiable checks:
\begin{enumerate}
\item Posterior stability under dataset subsets: BAO+SN; BAO+CMB; BAO+CMB+SN; and optional WL inclusion.
\item Stability under alternative coupling parameterizations (Eqs.~\eqref{eq:xi_cpl} vs \eqref{eq:xi_switch}).
\item Consistency between background-preferred regions and growth/lensing constraints.
\item Null recovery: $\xi\to 0$ reproduces \LCDM limits within the same pipeline.
\end{enumerate}

% =========================================================
\section{Mechanism-Level ``Smoking Gun'' Consistency Relations}
\label{sec:smokinggun}

A mechanism claim requires more than an improved fit to a subset of observables.
We propose the following mechanism-level targets, to be stated as explicit relations once the perturbation prescription is fixed:

\begin{itemize}
\item A relation linking the redshift of effective phantom crossing in \weff$(z)$ to a change in the slope of growth suppression in $f\sigma_8(z)$ residuals.
\item A coupled prediction relating growth residuals to CMB lensing residuals with shared dependence on $\xi(a)$.
\item A prediction for the sign and scale dependence of deviations in the growth index $\gamma$ relative to smooth \wz reconstructions.
\end{itemize}

These relations provide a route to falsification with near-term datasets (DESI full-shape, Euclid WL, improved CMB lensing).

% =========================================================
\section{Discussion: Status of Evidence and Interpretation}
\label{sec:discussion}

\subsection{Interpretation of DESI-era preferences}

Current joint analyses report a preference for dynamical dark energy at the level of a few standard deviations in some dataset combinations, with nontrivial dependence on priors and modeling choices.
Interacting dark-sector models can provide comparable or improved fits and, critically, can correlate background evolution with growth and lensing signatures.
At present, unique mechanism identification remains unsettled; DFM--MKC is proposed as a minimal candidate designed to be decisively tested via cross-probe consistency.

\subsection{Relation to other explanations}

We contrast DFM--MKC with:
(i) two-field quintom models, (ii) k-essence / EFT of DE parameterizations, and (iii) modified-gravity models.
The distinguishing feature of DFM--MKC is that coupling induces linked signatures across expansion and growth that can be formulated as testable consistency relations.

% =========================================================
\section{Conclusion}
\label{sec:conclusion}

We presented a minimal kinetic-coupling interacting dark-sector framework (DFM--MKC) that can reproduce effective \wz behavior compatible with DESI-era preferences while yielding coupled predictions for growth and lensing.
We outlined a robustness-centered inference plan and a falsification strategy aimed at converting phenomenological preference into mechanism-level confirmation or refutation with near-term data.

% =========================================================
\appendix

\section{Derivations and Stability Conditions}
\label{app:stability}

This appendix should include:
(i) the covariant interaction prescription $Q^\mu$, (ii) gauge conventions, (iii) the full linearized equations used in the Boltzmann/perturbation solver,
and (iv) explicit stability criteria (no ghosts / no gradient instabilities) for the parameter region explored.

\section{Implementation Notes}
\label{app:implementation}

Implementation guidance:
\begin{itemize}
\item CLASS/CAMB modification notes for background and perturbations under Eq.~\eqref{eq:Q_def}.
\item Validation by recovering \LCDM as $\xi\to 0$.
\item Unit tests for $H(z)$, $D_M(z)$, and growth observables against baseline pipelines.
\end{itemize}

% =========================================================
\bibliographystyle{apsrev4-2}
\bibliography{refs}

\end{document}

